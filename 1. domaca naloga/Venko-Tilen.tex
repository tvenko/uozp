% To je predloga za poročila o domačih nalogah pri predmetih, katerih
% nosilec je Blaž Zupan. Seveda lahko tudi dodaš kakšen nov, zanimiv
% in uporaben element, ki ga v tej predlogi (še) ni. Več o LaTeX-u izveš na
% spletu, na primer na http://tobi.oetiker.ch/lshort/lshort.pdf.
%
% To predlogo lahko spremeniš v PDF dokument s pomočjo programa
% pdflatex, ki je del standardne instalacije LaTeX programov.

\documentclass[a4paper,11pt]{article}
\usepackage{a4wide}
\usepackage{fullpage}
\usepackage[utf8x]{inputenc}
\usepackage[slovene]{babel}
\selectlanguage{slovene}
\usepackage[toc,page]{appendix}
\usepackage[pdftex]{graphicx} % za slike
\usepackage{setspace}
\usepackage{color}
\definecolor{light-gray}{gray}{0.95}
\usepackage{listings} % za vključevanje kode
\usepackage{hyperref}
\renewcommand{\baselinestretch}{1.2} % za boljšo berljivost večji razmak
\renewcommand{\appendixpagename}{Priloge}

\lstset{ % nastavitve za izpis kode, sem lahko tudi kaj dodaš/spremeniš
language=Python,
basicstyle=\footnotesize,
basicstyle=\ttfamily\footnotesize\setstretch{1},
backgroundcolor=\color{light-gray},
}

\title{Glasovanje za Pesem Evrovizije}
\author{Tilen Venko (63140280)}
\date{\today}

\begin{document}

\maketitle

\section{Uvod}

Cilj naloge je, da s pomočjo hierarhičnega razvrščanja preverimo, ali se pri glasovanju na Evroviziji ustvarjajo kakšne skupine držav, ki si medsebojno podeljujejo točke.

\section{Podatki}

Podatki so predstavljeni v dveh .csv datotekah. Ena s finalnimi in druga s polfinalnimi rezultati glasovanja. Pri preiskovanju sem uporabil samo podatke iz tabele s podatki o finalu, saj menim, da so polfinalni podatki manj zanesljivi in težje jih je obdelovati. \\
Podatki so zbrani za obdobje devetih let, od leta 1998 do leta 2009, za 46 držav. 

\section{Metode}

Naloge sem se lotil tako, da sem s csv.readerjem prebral podatke iz datoteke. Pobrisal nepotrebne podatke in si v slovar shranil imena držav, podatke o glasovanjih pa v transponirano tabelo. \\
Program vsebuje štiri ključne funkcije. In sicer funkcijo \textit{column\_distance}, ki izračuna evklidsko razdaljo med dvema stolpcema in jo uteženo vrne glede na to, koliko vrednosti smo uporabili. Države imajo namreč lahko različno število ujemanj v glasovanju med sabo. \\
Naslednja funkcija je \textit{cluster\_distance}, ki nad vsemi kombinacijami držav v gruči kliče funkcijo \textit{column\_distance} in vrne rezultat glede na parameter \textit{self.linkage}, ki je lahko \textit{complete, average ali single}.\\
Funkcija \textit{closest\_clusters} nad vsemi kombinacijami gruč kličemo funkcijo \textit{cluster\_distance} in vrnemo minimalni gruči.\\
Nato pa v funkciji \textit{run} v tabeli gruč združimo dve najbližji gruči. To počnemo tako dolgo, dokler nismo zadovoljni s številom gruč, ki nam jih program vrne (običajno 3). Te gruče na koncu še izpišemo.
\section{Rezultati}

rezultati ki nam mi jih vrne program za tri gruče so sledeči:

\begin{table}[]
\centering
\caption{Rezultati grupiranja}
\label{my-label}
\begin{tabular}{lll}\\\\
\textbf{1. skupina} & \textbf{2. skupina} & \textbf{3. skupina}    \\
Andorra             & Croatia             & Monaco                 \\
Portugal            & Montenegro          & Austria                \\
Spain               & Serbia              & Switzerland            \\
Romania             & Slovenia            & France                 \\
San Marino          & Malta               & Germany                \\
Bulgaria            & Ireland             & Belgium                \\
Cyprus              & United Kingdom      & Netherlands            \\
Greece              & Finland             & Albania                \\
Moldova             & Iceland             & Bosnia and Herzegovina \\
Czech Republic      & Sweden              & Macedonia              \\
Israel              & Denmark             & Serbia \& Montenegro   \\
Armenia             & Norway              &                        \\
Georgia             & Slovakia            &                        \\
Poland              & Estonia             &                        \\
Ukraine             & Latvia              &                        \\
Belarus             & Lithuania           &                        \\
Russia              &                     &                        \\
Turkey              &                     &                        \\
Azerbaijan          &                     &                        \\
Hungary             &                     &                       
\end{tabular}
\end{table}

\textit{Podatki niso v hierarhičnem drevesu, saj le-tega nisem znal naredit in sem podal samo države, ki so pogrupirane.}

\section{Izjava o izdelavi domače naloge}
Domačo nalogo in pripadajoče programe sem izdelal sam.


\end{document}
